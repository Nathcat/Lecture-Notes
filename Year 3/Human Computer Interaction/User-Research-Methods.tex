\section{User research methods}
\subsection{Interviews}
Types:
\begin{itemize}
	\item Unstructured --- Exploratory conversation with no script.
	\item Semi-structured --- Guide questions, but follow-ups allowed.
	\item Structured --- Tightly scripted with no deviation.
\end{itemize}

Pros:
\begin{itemize}
	\item Good for understanding wants, needs, and priorities.
	\item Versatile and flexible.
	\item Good for exploring issues in detail.
	\item Collect rich data.
	\item Can clarify questions and ask follow-ups.
\end{itemize}

Cons:
\begin{itemize}
	\item Less useful when users `don't know what they know.`.
	\item Difficult when users are shy.
	\item Expensive and time-consuming.
	\item People can be less likely to raise problems face to face.
\end{itemize}

\subsection{Focus groups}
Pros:
\begin{itemize}
	\item Can be an efficient way of hearing from more people.
	\item Good for community issues.
	\item Can help with ideas generation activities.
	\item Consensus can emerge.
	\item Can encourage users to talk more freely.
\end{itemize}

Cons:
\begin{itemize}
	\item Success depends on skill of facilitator.
	\item Less in-depth individual beliefs.
	\item Dominant participants can drown out others.
\end{itemize}

\subsection{Questionnaires}
Pros:
\begin{itemize}
	\item Can reach large numbers of people.
	\item Cost efficient.
	\item Can be quick.
	\item Easy to repeat.
\end{itemize}

Cons:
\begin{itemize}
	\item Limited opportunity for follow-up questions.
	\item May lack richness.
	\item Can't clarify / correct misunderstandings.
	\item People may give up halfway through.
\end{itemize}

\subsubsection{Good questions}
A good question can be open or closed. One should however make sure to ask about specifics.

Use open probes at the end, `do you want to tell me anything else?`. It is also good practice to summarise the interviewee's
responses.

\subsection{Observation}
Pros:
\begin{itemize}
	\item Can discover things that users can't express easily.
	\item Fills in details and allows you to get a rich picture.
	\item Helps you understand the context you are designing for.
\end{itemize}

Cons:
\begin{itemize}
	\item Can be costly and time-consuming.
	\item Potentially invasive.
	\item Can generate vast amounts of data.
\end{itemize}

\subsubsection{Indirect observation}
This approach might concern using user diaries, or probes. Users are asked to record their thoughts ideas / experiences
using some format of diary.

This can be used to gather information about their daily life, or their use of their device, for example.

Can deliver rich data, but you have little control over what is recorded, beyond initial prompts. It can also be disruptive / time
consuming for the user.

Pros:
\begin{itemize}
	\item Cost effective.
	\item Allows researchers to reach an extremely large number of users.
\end{itemize}

Cons:
\begin{itemize}
	\item The data gathered can lack context. We are told a lot about `what` and `when`, a little about `who` and `where`, but hardly anything about `why`.
\end{itemize}
