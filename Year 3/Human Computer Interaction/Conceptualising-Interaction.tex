\section{Conceptualising Interaction}
\begin{itemize}
	\item \textit{Assumption} --- Taking something for granted when it needs further investigation.
	\item \textit{Claim} --- Stating something is true when it is still open to investigation.
\end{itemize}

\subsection{Interface metaphors}
An interface metaphor is a conceptual model instantiated at the interaction interface. The interface should be designed
to be similar to physical entity, but which has its own properties, it could be based on an activity, object, or a combination of both.

They can be used in negative ways, for example:
\begin{itemize}
	\item Can break conventional and cultural rules
	\item Can constrain designers in the way they conceptualise a problem space.
	\item Can conflict with design principles.
	\item Can force users to only understand a system in terms of a metaphor.
	\item Designers might inadvertently use bad existing designs and transfer them over to a new design.
	\item Limit designers potential in coming up with new conceptual models.
\end{itemize}

\subsection{Instructions --- Conceptual model}
Instructions form a simple conceptual model, wherein the user issues commmands which are carried out by a system.
For example, a user might request that a computer save or open a file.

Commands might also be issued in multiple ways, for example through a GUI menu, or through a command line, etc...

The main benefit of this model is that it supports quick and efficient interaction, and can be particularly good for repetitive actions carried out on multiple objects.

Pros:
\begin{itemize}
	\item All commands are provided.
	\item Straightforward way of selecting options.
	\item Quick and efficient.
	\item Good for repetitive tasks.
\end{itemize}

Cons:
\begin{itemize}
	\item Limited number of commands.
	\item No flexibility.
	\item Need to learn the instructions.
\end{itemize}

\subsection{Conversing --- Conceptual model}
Underlying model of `having a conversation with a human`. This might range from simple voice recognition systems to more
complex natural language dialogues.

Some examples include timetables, search engines, advice-giving systems, and help systems.

Pros:
\begin{itemize}
	\item Designed to allow those unfamiliar / afraid of technology to interact with a system in an intuitive way.
	\item Hands free.
\end{itemize}

Cons:
\begin{itemize}
	\item Misunderstandings can arise when the system does not know how to parse what the human said.
	\item Can be annoying
	\item Unfulfilled user expections of system `intelligence`.
\end{itemize}

\subsection{Direct manipulation --- Conceptual model}
Involves dragging, selecting, opening, closing, and zooming actions on virtual objects. This approach exploits the user's
knowledge of how they move and manipulate the physical world. Can involve actions using physical controllers or air gestures
to control the movements of an on-screen avatar.

Pros:
\begin{itemize}
	\item Novices can learn the basic functionality quickly.
	\item Error messages are rarely needed.
	\item Users can get direct feedback on their actions.
\end{itemize}

Cons:
\begin{itemize}
	\item Some people take the metaphor of direct manipulation too literally.
	\item Not all actions can be done directly, and not all tasks can be described by objects.
	\item Can be slower than function keys or command line.
\end{itemize}

\subsection{Exploring --- Conceptual model}
Allows users to navigate through virtual or physical environments. 3D environments allow a more realistic exploration
experience.

Can also include physical environments with embedded sensor technologies --- Context aware technologies.

Pros:
\begin{itemize}
	\item Can be very useful when people don't know what they want.
	\item Good for games / entertainment where the exploration is part of the fun.
	\item Can provide a more immersive / engaging experience.
\end{itemize}

Cons:
\begin{itemize}
	\item Can take more time
	\item Requires very skilled design.
\end{itemize}
