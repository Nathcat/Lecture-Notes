\section{Designing Personas}
Peronas are user profiles which capture real characteristics.
They are not real people, but are synthesised from real user characteristics. Notably it is important
to not idealise a user's characteristics.

You might want to make them feel real, perhaps by giving them a name, characteristics, goals, a personal background, but it is important to avoid
stereotypes.

We use personas to help the designer make design decisions, to remind us that real people will be using the product, and to help
attempts in answering hypothetical questions about the product.

A persona determines what a prototype should do, and how it should do it, and can help different stakeholders communicate better. They can make it easier
to relate the prototype to the target user, compared to a list of features or flowcharts. They can help create empathy, and motivate the design
of a good UX. All in all, they are a good \textit{reality check tool}.

\subsection{User profiles}
A user profile is a brief biographical sketch. They are different to a persona.

A user profile may often contain information unrelated to the design task, and are likely based on stereotypes.

\subsection{Market segments}
Market segments play a role in persona development. They can help determine the demographic range, and can help understand the
relationship between the user and the markets. They are based on demographics, distribution channels, and purchasing behaviour, and therefore
no insights on user specific behaviour beyond this.

A market segment might be used in the research process, perhaps to build a pool of interviewees. The persona itself is then
derived from the results of this research.

\subsection{Demographics}
A demographic accoutns for the social background of the target user, it can account for the range ofa user's characteristics, \textit{but do not specify an average}.

In a big project, you would usually develop multiple personas representing key user types.

\subsection{Pitfalls in persona design}
One cannot create an \textit{elastic persona}, a single persona cannot describe an entire target audience.

\textit{Self-reflection design}, a persona cannot be based on you, it must be based specifically on the target audience,
not the people who designed it.

cool \neq good.

Try to avoid getting stuck on edge cases, one should always consider such cases, but do not focus solely on them.

\subsection{Constructing personas in 8 steps}
\subsubsection{1 --- Group interviewees by role}
These roles might include:
\begin{itemize}
	\item Job roles
	\item Family roles
	\item Attitudes to regular activities.
	\item Aptitudes regarding lifestyle choices.
\end{itemize}

\subsubsection{Identify behavioural variables}
\begin{itemize}
	\item Activities --- What the user does, frequency and volume.
	\item Attitudes --- How the user thinks about the technology.
	\item Aptitudes --- What education and training the user has.
	\item Motivations --- Why the user is engaged in the technology.
	\item Skills --- User abilities related to the technology.
\end{itemize}

\subsubsection{Map interviewees to behavioural variables}
Some variables will represent a continuous scale. Precise mapping onto this scale is not important,
you may simply rely on your gut feeling.

\subsubsection{Identify significant behavioural patterns}
Look for clusters of interviewees, these behavioural patterns will form the basis of your persona.

Specifically meaningul patterns.

\subsubsection{Synthesise characteristics and define goals}
For each behaviour pattern, define the following:
\begin{itemize}
	\item The behaviours themselves.
	\item The use environments
	\item Frustrations and pain points related to the behaviour
	\item Demographics associate with the behaviour
	\item Skills, experience, or abilities relating to the behaviour.
	\item Attitudes and emotions associated with the behaviour.
	\item Relevant interactions with other people.
	\item Alternate or competing ways of doing the same thing.
\end{itemize}

\subsubsection{Check for completeness and redundancy}
Check for gaps in your persona template, are there any behaviours missing?
Eliminate any redundant personas, if you have any.

The final persona should be as compact as possible.

\subsubsection{Designate persona types}
\begin{itemize}
	\item Primary (the main target of the interface design)
	\item Secondary (may have specific additional needs)
	\item Supplemental (could be added due to political reasons)
	\item Customer (a customer in the segment, but not an end user)
	\item Served (do not use the prototype but are affected by it)
	\item Negative (someone you are not designing for)
\end{itemize}

\subsubsection{Expand the description of attributes and behaviours}
Summmarising descriptions of significant behaviours, do not include excessive fictional details, do not add what
you did not observe, do not include solutions, highlight only frustrations.

Precise and credible details are vital to making a persona look and feel like a real end user.

You might include a photo of the user you describe in your persona.

\subsection{Secondary research based personas}
Best viewed as \textit{proto personas}, and can serve as a good starting point. They should be used for future validation
of personas created from primary research.

One must clearly indicate which elements are derived from the secondary persona, and which are creatively interpreted.
