\section{Establishing requirements}
Requirements are statements which specify what a product should do, and how it should do it. Establishing requirements is the
stage in which most project failure occur, so it is important that it is done correctly.

\subsection{Functional and non-functional requirements}
A functinal requirement describes what a system should do, while a non-functional requirement describes how it should
be done.

Functional requirements of a mobile phone:
\begin{itemize}
	\item Calling
	\item Texting
	\item Emailing
	\item Phtography
	\item Noise activation
	\item Notifications
\end{itemize}

Non-functional requirements of a mobile phone:
\begin{itemize}
	\item Stable access to a network
	\item Good battery life
	\item Robust bandwidth
	\item Sufficient storage
\end{itemize}

\subection{Contextual requirements}
These describe relationships or dependencies between sets of objects in a system.

These are objects that need to be displayed together to make sense for a specific workflow, or to meet the goals of a persona.
They may include considerations of a physical environment, and the capabilities / skills of a user.

\subsection{User requirements}
Who will be using the system, are they a novice, expert, frequent, casual / infrequent user?

\subsection{Usability / UX requirements}
Usability requirements are based on usability goals such as ease of use, learnability, efficiency, safety, and
other associative measures.

UX requirements are based on UX goals such as fun, enjoyable, entertaining. These are usually much harder to specify in terms
of requirements.

\subsection{Design requirements}
Design requirements specify the what instead of the how, and should be defined before the solutions.

\begin{enumerate}
	\item Create problem and vision statements.
	\item Explore and brainstorm
	\item Identify persona expectations
	\item Construct context scenarios
	\item Identify design requirements
\end{enumerate}


