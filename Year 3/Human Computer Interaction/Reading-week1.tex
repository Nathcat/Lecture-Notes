\section{Reading Notes --- Week 1}

\textit{Beyond Human Computer Interaction --- 5th Edition}, Chapter 1

HCI in practice requires a lot of diversity in a team to make work, depending on the user scope of the end product.
A benefit of this is including people from a number of different backgrounds in your team, which leads to new ideas.
It can however be hard for people of such varied backgrounds to come together in such a way. Having to account for
such a variety in backgrounds and organizational structures, the more complex the team and the development process is
likely to be.

The user experience refers to how a product behaves and is used by people in the real world. Every product which might
be used by a person has a user experience attached to it, this experience is defined by how people feel about a product
and their pleasure and satisfaction when using it. An important point is that one cannot directly design a user 
experience, but only create the design features that give rise to it. The experience itself is determined by the user
themselves.

Being aware of people's sensitivities, such as ageing, is as important as knowing how to design for their
capabilities. Furthermore, an awareness of cultural differences is also an important concern, dependent of course on the
intended scope of the product.

One should maintain a focus on accessibility, because a product that is designed with accessibility in mind can be used
by anyone. 

The distinction between usability goals and user experience goals is not clear-cut, since usability is often fundamental
to the quality of the user experience, i.e. poor usability generally leads to a poor experience.

Usability is generally broken down into the following six goals:
\begin{itemize}
	\item Effectiveness
	\item Efficiency
	\item Safety
	\item Utility
	\item Learnability
	\item Memorability
\end{itemize}

When learning how best to improve the usability of a system, one should not simply ask where the areas for improvement
are ("Is the system easy to learn?"), one should instead ask more targeted questions, for example: "How long will it take a user to figure out how to use the most basic functions for a new smartwatch; how much can they capitalise on from
their prior experience; and how long would it take the user to learn the whole set of functions?".

One may also make use of quantitative metrics to improve the usability of a system. For example:
\begin{itemize}
	\item Time to complete a task (efficiency)
	\item Time to learn a taskk (learnability)
	\item The number of errors made when carrying out a given task over time (memorability)
\end{itemize}

User experience goals conversely, are defined in terms of the feelings a user might have when using a system. Said
feelings might also be further defined in terms of elements that contribute to making a user experience pleasurable,
fun, exciting, and so on. Examples might be attention, pace, play, interactivity, conscious and unconscious control,
style of narrative, and flow.

The quality of a user experience may also be affected by single actions performed at an interface, known as 
micro-interactions, such as the sounds of trash being emptied from the trashcan on a screen. Although small, these 
micro-interactions can have a big impact on user experience.
