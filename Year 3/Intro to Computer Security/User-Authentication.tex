\section{User Authentication}
There are four factors which may be used when authenticating a user's identity:
\begin{itemize}
	\item Something the individual knows, i.e.\ some password / PIN.
	\item Something the individual possesses, i.e.\  A token, smartcard, keycard, etc...
	\item Something the individual is, i.e.\ Biometrics
	\item Something the individial does, i.e.\ Voice pattern, handwriting, typing rhythm
\end{itemize}

\subsection{Assurance Levels}
The \textit{Assurance Level} of an authentication system describes an organisation's
degree of certainty that a user has presented a credential which refers to his or her unique
identity. There are generally assumed to be four levels of assurance, with 1 being the lowest
confidence, and 4 being the highest.

\subsection{Potential Impact of security breaches}
FIPS 199 defines three levels of potential impact on organisations / individuals should there
be a breach of security. These three levels are Low, Moderate, and High.

The assurance level of a system is directly tied into its potential impact, i.e.\ the
potential impact of a breach in a system should be lower in conjunction with the assurance
level. The lower assurance levels are more likely to break, and therefore should be linked
to systems which have less potential impact. Systems with higher assurance levels should
be used for more critical systems.

\subsection{Password Authenticaton}
This method of authentication requires a passphrase, which theoretically is known only
to the user. Thus ensuring their identity.

Passwords will need to be stored in the authentication system in order to verify their
correctness, but it is important to note that they should not be stored in a plaintext
format. The accepted method to store passwords is as a secure hash. The user would then
send their plaintext password to the authentication server, which will hash the entry with
the same algorithm, and compare the result to the hash it has for that user.

Password addition rules can help reduce the likelihood of a dictionary attack, should an
attacker manage to obtain hashed passwords. A salt will also help mitigate this risk by 
introducing entropy into the stored hash.

Note that when using salting, the salt used will need to be stored along with the hashed value
in some way, otherwise you will not be able to compare the plaintext against the salted hash.
Algorithms such as BCrypt do this for you, by including the hash as part of the outputted hash.

\subsubsection{Password Cracking}
\begin{itemize}
	\item Dictionary attacks --- A dictionary of possible passwords is hashed and then compared against the stored hashes.
	\item Rainbow table attacks --- Use pre-computed tables of hash values for all hashed passwords, lookup the stored hash in this table. This can be countered with a sufficiently large salt and hash length.
\end{itemize}

Password crackers such as John the Ripper use the above techniques and exploit the fact that people tend to
choose easuly guessable passwords.

A lot of these techniques, although they have improved with computational resources, can be fended off through
a number of approaches:
\begin{itemize}
	\item User education
	\item Computer generated passwords and password managers
	\item Reactive password checking
	\item Complex password policy
\end{itemize}

\subsection{Memory Cards}
These are cards which store data, but do not process data. The most common is the magnetic stripe card.

The intention is that they can store some internal electronic memory, which is used for physical access to something,
such as a Hotel room or ATM. The security of this approach is improved when combined with a password or a PIN.

They do however require a special reader, and suffer from potential loss of the card itself.

\subsection{Smart Tokens}
These include an embedded microprocessor, and potentially a keypad / display for human interaction.

Requires some elecrtonic interface to communicate with, may be either contact or contactless. They generally
provide authentication through one of three strategies, Static, Dynamic password generator, and Challenge-response.

\subsection{Electronic Identity Cards}
These are smart cards used as a national identity card for citizens. They can provide a stronger proof of identity and
can be used in a wider variety of applications.

\subsection{SmartPhones}
Smartphones provide a common interface for 2FA applications.

\subsection{Biometric authentication}
This approach verifies an individual's identity based on unique physical characteristics, such as a fingerprint.

\subsection{Remote user authentication}
This is a concept which provides authentication over the internet, or a remote communications link. It is of a higher risk
and therefore must be more complex to ensure safety of information and identity.

There are additional security threats such as eavesdropping, password capturing, replaying an observed authentication sequence, etc\ldots

Generally this method relies on some form of challenge-response protocol to counter threats.

\subsection{Federated Identity Management}
This is a relatively new concept involving the use of a common identity management scheme across
multiple applications which support many users.

The central idea is the use of a Single-Sign-On protocol / service.