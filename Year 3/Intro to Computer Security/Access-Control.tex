\section{Access Control}
In a broad sense, all of Computer Security is simply a concern of access control,
RFC 4949 defines computer security as `measures that implement and assure security
services in a computer system, particularly those that assure access control service'.

\subsection{Policies}
\subsubsection{Discretionary access control (DAC)}
Controls access based on the identity of the requestor and on access rules
stating what requestors are (or are not) allowed to do.

Under this scheme an entity may be granted access rights that permit the entity, by its own volition,
to enable another entity to access some resource. It is usually provided using an \textit{Access control matrix}.

In such a matrix, one dimension consists of identified subjects that may attempt to access
the resources, while the other dimension lists the objects that may be accessed. Each entry in the matrix then consists of
a particular access right setting, for a particular subject and object.

\subsubsection{Mandatory access control (MAC)}
Controls acess based on computing security labels with clearances.

\subsubsection{Role-based access control (RBAC)}
Controls access based on the roles that users have within the system and on rule stating 
what accesses are allowed to users in given roles.

\subsubsection{Attribute-based access control (ABAC)}
Controls access based on attributes of the user, the resource to be accessed, and the 
current environmental conditions.

This approached evaluates rules against the attributes of entities, operations, and the environment
which is relevant to a request. These attributes must be evaluated and compared against access control rules defining
subject / object attribute combinations in a given envrionment.

\subsection{UNIX File Access Control}
On the disk of a UNIX system is an inode table, which contains inodes which can be associated
with several file names, although once active, an inode is loaded into memory and is then associated with only one file.

An Inode contains control structires with key informatin needed for a particular file.

Directories are structured in a hierarchical tree, each may contain file / subdirectories, and pointers to associated inodes.

Each object has 12 protection bits attached to it, the first three for a single user designated as the owner, the second set for the group,
and the third for others. The remaining 3 bits are for other flags, which are often not used and as such most commonly set to 0.
The owner, group ID, and protection bits are part of the file's inode.

SetUID, and SetGID, these flags allow the system to temporarily use the rights of the file owner / group in addition to the real
user's rightrs. This enables privilged programs to access files / resources which are not generally accessible.

The sticky bit is applied to a directory, it specifies that only the owner of any file in the directory can rename, move, or delete that file.

The superuser however is exempt from access control restrictions and has system-wide access.

An Access Control List assigns a list of user IDs and groups to a file, each with their own set of protection bits.
A file does not necessarily need to have an ACL, and an additional protection bit must be included to indicate whether the file has an
extended ACL.