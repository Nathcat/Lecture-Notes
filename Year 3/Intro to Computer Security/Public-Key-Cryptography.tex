\section{Public Key Cryptography}
Public key cryptography using a mathematically related set of two keys,
one public and one private, to encrypt and decrypt data.

Data is encrypted using a public key, which is freely available to anyone,
while private keys are used to decrypt ciphertext, and should be kept secret.

Public-key cryptosystems have a set of requirements which must be
met by an algorithm to be viable:
\begin{itemize}
    \item Computationally easy to create key pairs.
    \item Useful if either key can be used for each role.
    \item Computationally infeasible for opponent to otherwise recover original message.
    \item Computationally infeasible for opponent to determine private key from public key
    \item Computationally easy for receiver, knowing private key, to decrypt ciphertext.
    \item Computationally easy for sender, knowing public key, to encrypt messages
\end{itemize}

Some examples of asymmetric encryption algorithms:
\begin{itemize}
    \item RSA
    \item Diffie Hellman key exchange --- Enables two users to agree on a shared secret which can be used in subsequent symmetric encryption
    \item Digital Signature Standard (DSS) --- Only provides a digital signature function with SHA-1, cannot be used for encryption or key exchange.
    \item Elliptic Curve Cryptography --- Like RSA, but with much simpler keys.
\end{itemize}


\subsection{Digital Signatures}
A digital signature is defined by NIST as `The result ofa cryptographic transformation of data that,
when properly implemented, provides a mechanism for verifying origin authentication, data integrity, and
signatory non-repudiation'.

A digital signature therefore is a data-dependent bit-pattern, generated by some agent as a function of some data block.
\begin{itemize}
    \item DSA
    \item RSA Digital Signature Algorithm
    \item ECDSA
\end{itemize}

A digital signature algorithm generally accepts the hash of the data block, and the agent's private key,
using these it can provide a bit pattern signature.

\subsection{Public Key Certificate}
A Public key certificate contains user ID, a public key, and inforamtion about a CA.
This information is passed through a DS algorithm, with the CA's private key, producing a signed
public key certificate.

The signature can be verified using the CA's public key. The signature is decrypted and the certificate hashes compared,
to determine whether or not the signature is valid.

\subsection{Digital Envelope}
A digital envelope contains a message encrypted using a symmetric key, and said symmetric key, encrypted
using the receiver's public asymmetric key.

This process effectively combines both symmetric and asymmetric encryption layers.

\subsection{Random Numbers}
Random numbers are used in the generation of all kinds of keys, asymmetric, stream, and symmetric. They are also used in handshaking to
prevent replay attacks, and in session key creation.

These cryptographic functions however have certain requirements to the type of random numbers they can use.
\begin{itemize}
    \item Uniform distribution --- The frequency of occurrence of each of the numbers should be approximately the same.
    \item Independence --- No one value in the sequence can be inferred from the others.
\end{itemize}

Cryptographic applications typcially make use of algorithmic techniques to produce pseudorandom numbers. Such numbers are
part of sequences produced to satisfy statistical randomness tests, and because of this are much more likely to be predicable.

A true random number generator is much more difficult to create, there must be some nondeterministic source of entropy to produce true randomnes.
Most such applications operate by measuring unpredicable natural processes, and are being increasingly included in the hardware of modern processors.