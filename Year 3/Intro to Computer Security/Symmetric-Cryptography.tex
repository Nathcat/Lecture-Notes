\section{Symmetric Cryptography}

Symmetric Encryption is a universal technique for providing confidentiality to transmitted or stored data.
It is also referred to as conventional, and single key encryption. It generally requires a strong encryption 
algorithm to be viable, and that both parties have a copy of the secret key.

Since it only requires one key, this one key must be kept and transmitted securely.


\subsection{Attack Symmetric Encryption}

\subsubsection{Cryptanalytic attacks}
Cryptoanalytic attacks rely on knowing the nature of the underlying algorithm, and some knowledge of the general
characteristics of the plaintext, potentially with some ciphertext / plaintext pairs.

The idea is to use this knowledge to deduce the key which has been used, thus allowing the attacker to gain
access to any message encoded using this key.

\subsubsection{Brute Force attacks}
As the name suggests, this approach involves trying all possible keys until one which produces a resonable plaintext
is found. On average of course this means trying half of all possible keys.

Logically, this method becomes progressively more unfeasible as the key size increases.


\subsection{Data Encryption Standard}
\textit{DES} was one of the first widely used encryption schemes, published in 1977.

It is a symmetric block cipher, which encodes 64 bit blocks of data using a 56 bit key. The fact that this is one of the
most studied encryption algorithms in existence, and the small key size, leads to concerns over its security. It is in fact,
remarkably easy to break.

\subsection{3DES}
3DES is an improvement over DES.\ Internally it uses the same algorithm for encryption, but it uses said
algorithm three times, with three unique keys. This means a key size of 168 bits, which massively increases
the security of the algorithm, making it unfeasible to brute force it.

The repetition and small block size however, means that this algorithm is slow and inefficient on larger data sets.

\subsection{AES}
In 1997, NIST called for an improvement to 3DES.\ The selected proposal was the Rijdael Cipher, now known as the 
\textit{Advanced Encryption Standard (AES)}, this algorithm has an improved security over 3DES, and is more efficient.

It has the options for 128 / 192 / 256 bit keys, and can operate on 128 bit blocks.

\subsection{Practical Issues}
Symmetric encryption is usually applied to blocks of data larger than 64 or 128 bits, this means that the plaintext
must be split up into blocks of this size and processed as such.

Electronic Codebook (ECB) is the simplest approach to this, wherein each block of data is simply encrypted with the same
key. While simple and easily paralleled, this approach can introduce regularities into the ciphertext, which can be
exploited in Cryptanalysis.

Two solutions to this are cipher block chaining, and stream ciphers.

\subsubsection{Cipher Block Chaining}
In this approach, one block is processed at a time, with the output of the current block being XORed into the next block,
starting with an initialisation vector.

This leads to a reduction in regularities in the ciphertext.

\subsubsection{Stream Ciphers}
In Stream Ciphers, input elements are processed continuously, and the output is produced one element at a time. The
key for each element is produced by some Pseudorandom stream which is generated by an initial secret key.

This method is both fast, and simple, and is therefore ideal for real-time applications like video streaming. Its
Pseudorandom nature also makes it harder to break through Cryptanalysis.

\subsection{Message authentication}
While Symmetric Cryptography ensures the confidentiality of data, it does not ensure the authenticity and integrity of the
sender.

Introducing authentication protects against active attacks to a system, by verifying if the received message came from
the expected source, and that it has not been tampered with (integrity). Conventional cryptographic algorithms can be
made to ensure authenticity by including some authentication tag in the message, but typically it is provided through
some other means.

\subsubsection{Cryptographic hash function}
Many authentication algorithms use one-way hash functions. These functions encrypt variable size blocks of data
into a fixed size buffer, which cannot be used to obtain the original plaintext (hence one-way).

By including some hashed token at the end of an encrypted message, the receiver can verify that the hash of this
token matches what they expecte from the sender, thus ensuring authenticity.

If the token is in some way related to the state of the message, this approach might also infer that the message's
integrity is intact.