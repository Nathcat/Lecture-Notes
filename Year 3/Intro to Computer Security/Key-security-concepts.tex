\section{Key Security Concepts}

\begin{itemize}
	\item Confidentiality --- Preserving authorized restrictions on access, including means for protecting personal privacy and proprietary information.
	\item Integrity --- Guarding against improper information modification or destruction.
	\item Availability --- Ensuring timely and reliable access to and use of information.
	\item Authenticity --- Assuring confidence in the validity of a transmission, message, or message ordinator.
	\item Accountability --- Providing the capability of actions being traced to their originator. Records should be kept to provide post-attack analysis and for meeting legal requirements.
\end{itemize}

\subsection{Vulnerabilities, Threats, and Attacks}
\subsubsection{Vulnerabilities}

\begin{itemize}
	\item Corruption --- loss of integrity.
	\item Leaky --- loss of confidentiality.
	\item Unavailable or very slow --- loss of availability.
\end{itemize}


\subsubsection{Threats}

\begin{itemize}
	\item Capable of exploiting vulnerabilities.
	\item Represent potential security harm to an asset.
\end{itemize}


\subsubsection{Attacks --- Threats carried out}

\begin{itemize}
	\item Passive --- Attempt to learn or make use of information from the system that does not affect system resources.
	\item Active --- Attempt to alter system resources or affect their operation.
	\item Insider --- Initiated by an entity within the security perimeter.
	\item Outsider --- Initiated from outside the perimeter.
\end{itemize}

Countermeasures are means to deal with attacks, their goal is to prevent, detect, and recover from attacks. They are good at dealing with attacks, but they do not correct 
or rectify the underlying vulnerability that gave rise to the initial threat. Their intent is to minimize the risk, rather than prevent it entirely.
