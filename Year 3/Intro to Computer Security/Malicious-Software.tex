\section{Malicious Software}
Malware is classified into two broad categories, based firstly on how it
spreads and propogates to reach its desired targets, and then on the actions
or payloads it performas once a target is reached.

They can also be classified by those that need a host program; those that
are independent, self contained programs; Malware that does not replicate;
and Malware that does replicate.

Some propogation mechanisms are:
\begin{itemize}
    \item Infection of existing content by viruses that are subsequently spread to other systems
    \item Exploit of software vulnerabilities by worms or drive-by-downloads to allow the malware to replicate
    \item Social engineering attacks that convince users to bypass security mechanisms to install Trojans or respond to phishing attacks.
\end{itemize}

Payload actions performed by malware can include:
\begin{itemize}
    \item Corruption of system or data files
    \item Theft of service/make the system a zombie agent of attack as part of a botnet
    \item Theft of information from the system / keylogging
    \item Stealthing / hiding its presence on the system
\end{itemize}

\subsection{Attack kits}
Initially, the development and deployment of malware required considerable technical skill
by software authors, but the development of virus-creation toolkits in the 1990s and the more general
attack kits in the 2000s greatly assisted in the development and deployment of malware.

These toolkits are often known as `crimeware'. They include a variety of propagation mechanisms and payload modules
that even novices can deploy. Variants that can be generated by attackers using these toolkits creates a significant
problem for those defending systems against them.

\subsection{Attack sources}
\begin{itemize}
    \item Politically motivated attackers
    \item Criminals
    \item Organised crime
    \item Organisations that sell their services to companies and nations
    \item National government agencies
\end{itemize}

A significant change in malware development is the shift from attackers being individuals to more organised groups.

\subsection{Advanced Persistent Threats}
This is a well-resourced, persistent applicaiton of a wide variety of intrusion technologies and malware to selected targets.
This kind of attack is typically attributed to state-sponsored organisations and criminal enterprises.

They differ from other types of attack by their careful target selection and stealthy intrusion effors over extended periods.

The intent is to infect that target with sophisticated malware with multiple propagation methods and payloads, and once infected, use
a further range of attack tools to maintain and extend their access.

\subsection{Viruses}
This is a simple piece if software which infects programs. The program is modified to include a copy of the virus, which is then replicated and goes on to infect
other content. This kind of software can easily spread through network environments.

When attached to an executable program, a virus is able to do anything that the program is permitted to do.

\subsubsection{Virus phases}
\begin{enumerate}
    \item Dormant phase --- Virus will be activated by some future event. Not all viruses have this stage.
    \item Triggering phase --- Virus is activated to perform the function for which it was intended.
    \item Propagation phases --- Virus places a copy of itself into other programs or into cetrain system areas on the disk.
    \item Execution phase --- Function is performed, this may be harmless or damaging.
\end{enumerate}

\subsubsection{Macro and Scripting Viruses}
Macro viruses infect scripting code used to support active content in a variety of user document types.
They are platform independent, and aim to infect documents rather than executable code. Because of this they are easily spread and can be very threatening.

\subsubsection{Virus classifications}
Classification by target:
\begin{itemize}
    \item Boot sector infector --- Infects a master boot record and spreads when the system is booted from the disk containing the virus.
    \item File infector --- Infects files that the operating system or shell considers to be executable.
    \item Macro virus --- Infects files with macro or scripting code that is interpreted by an application.
    \item Multipartite virus --- Infects files in multiple ways.
\end{itemize}

Classification by concealment strategy:
\begin{itemize}
    \item Encrypted virus --- A portion of the virus creates a random encryption key and encrypts the remainder of the virus.
    \item Stealth virus --- A form of virus explicitly desgined to hide itself from detecion by anti-virus software.
    \item Polymorphic virus --- A virus that mutates itself with every infection.
    \item Metamorphic virus --- A virus that mutates and rewrites itself completely at each iteration, and may change behaviour as well as appearance.
\end{itemize}


\subsection{Worms}
A worm is a program that actively seeks aout more machines to infect and each infected machine serves as an automated launching pad for attacks on other machines.

This approach exploits software vulnerabilities in client or server programs. They can use network connections to spread from system to system.
They can spread through shared media such as USB drives, CD, etc\ldots.

Email worms sprad in macro or script code included in attachments and instant messenger file transfers.

Upon activation the worm may replicate and propagate again, and they usually carry some form of payload.

\subsubsection{Worm replication}
\begin{itemize}
    \item Electronic mail or instant messengers --- Worm copies itself as an attachment.
    \item File sharing --- Creates a copy of itself or infects a file as a virus on removable media.
    \item Remote execution capability --- Worm executes a copy of itself on another system.
    \item Remote file access or transfer capability --- Worm uses a remote file access or transfer service to copy itself from one system to the other.
    \item Remote login capability --- Worm logs onto a remote system as a user and then uses commands to copy itself from one system to another.
\end{itemize}

