\documentclass[a4paper]{article}
\usepackage{amsmath}
\usepackage{amsfonts}
\usepackage{amssymb}
\usepackage{hyperref}
\hypersetup{
    colorlinks=true,
    linkcolor=blue,
    filecolor=magenta,      
    urlcolor=cyan,
    pdftitle={Overleaf Example},
    pdfpagemode=FullScreen,
    }

\usepackage{listings}
\usepackage{color}
\definecolor{dkgreen}{rgb}{0, 0.6, 0}
\definecolor{gray}{rgb}{0.5, 0.5, 0.5}
\definecolor{mauve}{rgb}{0.58, 0, 0.82}

\newcommand{\norm}[1]{\left\lVert#1\right\rVert}

\title{G5015 --- Comparative Programming}
\author{Dr. James Bennet \and Notes by Nathan Baines}
\date{Academic year 2025/26}

\begin{document}
\maketitle

\section{Linear Algebra}

\subsection{Linear models}
A linear model describes a relationship between one variable and another.
The most common linear model you will be familiar with is:

\begin{equation}\label{straight-line}
	y = mx + c
\end{equation}

In GCSE Maths, this is referred to as \textit{the equation of a straight line}. It describes a
relationship between two variables, in this case $x$ and $y$, by using two constants, $m$, and $c$.
i.e. we are predicting the behaviour of the variable $y$ as $x$ changes.

There are of course, many many other kinds of linear model, the only real constraint is that they describe
a relationship between two variables.

A more formal notation:

\begin{equation}
	\hat{f}(x) = w_0+w_1x
\end{equation}

This describes the same model as \autoref{straight-line}, simply in terms of a function $\hat{f}$.

\subsection{Vectors}

Vectors are a way of describing variables which contain multiple dimensions. It is commonly said
that they have both direction and magnitude, while \textit{scalar} values have only magnitude.

They can be used in many scenarios, but the most common is probably to describe a point, or a transformation in
an $N$ dimensional space.

A key point is that vectors are effectively matrices in the context of linear algebra. By which I mean they
are presented in the same way, and can be combined (and often are) in the same calculations one might find
matrices.

The main difference between a vector and a matrix, is that where matrices can have any number of columns or rows,
a vector will have either one row, or one column.

The shape of a matrix is often given by:
\begin{equation}
	\textrm{rows}\times \textrm{columns}
\end{equation}

... so while a matrix might be $M\times N$, a vector will either be $1\times N$, or $N\times 1$.

If a system makes use of vectors which are presented as a row ($1\times N$), they are called \textit{row major systems}.
It follows then that a system which presents vectors as columns ($N\times 1$), are called \textit{column major systems}.

\subsubsection{Operations with vectors}
Vectors follow the same operational rules as matrices, but they have some special operations.

The magnitude of a vector with $N$ dimensions, is given by:
\begin{equation}\label{vector-magnitude}
	|\vec{a}| = \sqrt{\sum^N_{n=1}\vec{a}_n^2}
\end{equation}

A vector $\vec{a}$ is a unit vector, if it satisfies $|\vec{a}|=1$. The unit vector, or simply put, the direction
of a vector, can be found by \textit{normalising} said vector:

\begin{equation}\label{vector-normalisation}\begin{split}
		& \textrm{Let }\vec{b}\textrm{ be a vector with magnitude }>1 \\
		& \hat{\vec{b}} = \frac{\vec{b}}{|\vec{b}|}
	\end{split}\end{equation}

Note that one may perform the above division, simply by dividing all the elements of vector dividend, by the scalar divisor.

The dot product is \textit{effectively} a measure of how different two vectors $\vec{a},\vec{b}$ are. (Note that
$\vec{a}$ is notation for a vector named $a$). It forms a relation between the components of the vector,
and the \textit{acute} angle between them $\theta$.

\begin{equation}\label{dot-product}\begin{split}
		& \textrm{Let }N\textrm{ be the number of dimensions} \\
		\vec{a}\cdot\vec{b} & \\
		&= |\vec{a}|\times|\vec{b}|\times\cos(\theta) \\
		&= \sum^N_{n=1}\vec{a}_n\vec{b}_n
	\end{split}\end{equation}

The cross product, allows you to find the vector which is perpendicular to two vectors. Note that $\vec{n}$
is the unit vector perpendicular to the two vectors $\vec{a}$ and $\vec{b}$.
\begin{equation}\label{cross-product}\begin{split}
		\vec{a}\times\vec{b} & \\
		&= |\vec{a}|\times|\vec{b}|\times\sin(\theta)\times\vec{n}
	\end{split}\end{equation}

You may ask youself, `but Nathan, that's stupid! You're saying that to find the perpendicular vector to two vectors,
you need the perpendicular unit vector!`, and you would be correct. The cross product is, for reasons I cannot
be asked to understand or explain, only defined to be non-zero in three and seven dimensions. As such, we have the
following if both $\vec{a}$ and $\vec{b}$ have three dimensions:

\begin{equation}
	\vec{a}\times\vec{b} = \left[\begin{matrix}
			\vec{a}_y\vec{b}_z - \vec{a}_z\vec{b}_y \\
			\vec{a}_z\vec{b}_x - \vec{a}_x\vec{b}_z \\
			\vec{a}_x\vec{b}_y - \vec{a}_y\vec{b}_x
		\end{matrix}\right]
\end{equation}



\appendix
\section{Lab 2 --- Question 2}
We found this question quite challenging in my group, so after having managed to prove it, I will display said proof here.

The question was as follows:
\begin{equation}\begin{split}
		&\textrm{The classification boundary of a perceptron model is given by}                    \\
		&\mathbf{w}^T\mathbf{x}+b=0                                                                \\
		&\textrm{Show that the shortest distance between this boundary and the origin is given by} \\
		&\frac{b}{\norm{\mathbf{w}}}
	\end{split}\end{equation}

Firstly, it is worth noting that the expression given for the boundary, is a simple euclidean form line equation, represented using vectors.
$\mathbf{w}$ is a vector containing the coefficients of the equation, and $\mathbf{x}$ contains the variables of the equation.

Let us suppose, that this is a simple $y=mx+c$ form equation, if this is to be the case then we must define the following for $\mathbf{w}$ and $\mathbf{x}$.
\begin{equation}
	\mathbf{w}=\left(\begin{matrix}w_0 \\ w_1\end{matrix}\right),
	\mathbf{x}=\left(\begin{matrix}x \\ y\end{matrix}\right)
\end{equation}

We can then see that,
\begin{equation}\begin{split}
		\mathbf{w}^T\mathbf{x}+b=0 &\to \left(\begin{matrix}w_0 & w_1\end{matrix}\right)\left(\begin{matrix}x \\ y\end{matrix}\right)+b=0 \\
		&\to w_0x + w_1y + b = 0 \\
		&\to w_1y = -w_0x-b \\
		&\to y = -\frac{w_0}{w_1}x-\frac{b}{w_1}
	\end{split}\end{equation}

This is now in a more obvious form of the euclidean line equation ($m=-\frac{w_0}{w_1}$, $c=-\frac{b}{w_1}$), a form we can work with much more easily.

The aim of the question is to find the shortest distance between this line, and the origin $(0, 0)$. This can be found by
finding the perpendicular line which passes through \textit{both} our boundary, and the origin, then finding the point
where this perpendicular line intercepts the boundary, and then finding the distance of that point from the origin, which
will simply be the euclidean norm: $\sqrt{x^2+y^2}$.

It is known that for a line $y=mx+c$, a perpendicular line passing through said line would be of the form $y=-\frac{1}{m}x+c_1$.
Remember that $c$ indicates the point at which a line passes through the y-axis (the y-intercept), and $m$ is the gradient of the line.

Given this knowledge, we can see that the line perpendicular to our boundary must have a gradient of $\frac{w_1}{w_0}$, i.e. $m=\frac{w_1}{w_0}$.

\begin{equation}
	y_p=\frac{w_1}{w_0}x+c
\end{equation}

Note that I am using $y_p$ to refer to perpendicular line, this is not a commonly used notation (to my knowledge), just something I like
to do to make my work a little clearer.

Now we must find $c$, the y-intercept. Remember that this line \textit{must pass through the origin}, so at some point along it, we will have
$y=0$, and $x=0$, so let us substitute these values, and solve for $c$:

\begin{equation}
	0=\frac{w_1}{w_0}(0)+c \to c=0
\end{equation}

We have $c$. Note that $c=0$ is commonplace when a line must pass through the origin, since it intercepts the y-axis at $y=0$.

Given what we have found, we can now see that the perpendicular line to our boundary which passes through the origin has the euclidean form:
\begin{equation}
	y=\frac{w_1}{w_0}x
\end{equation}

The next step, is to find the point where this line intercepts our boundary line, and for this, we will approach the problem with simultaneous equations.

When the lines intercept, they will have the same values of $x$ and $y$, so we can begin by simply setting them equal to each other (since they are equal
at the point of intercept!), and solving for $x$. We solve for $x$, since this will give us the $x$ ordinate of their intercept point. We can then substitute
that value back into one of the original line equations to find the $y$ ordinate.

\begin{equation}\begin{split}
		\frac{w_1}{w_0}x &= -\frac{w_0}{w_1}x-\frac{b}{w_1} \\
		\to \frac{w_1^2}{w_0}x &= -w_0x-b \\
		\to \frac{w_1^2}{w_0}x+w_0x &= -b \\
		\to \frac{w_1^2+w_0^2}{w_0}x &= -b \\
		\to x &= \frac{-bw_0}{w_0^2+w_1^2} \\ \\
		&\textrm{Now we substitute back into our original equations to find }y \\
		y &= \frac{w_1}{w_0}x \\
		\to y &= \frac{w_1}{w_0}\cdot\frac{-bw_0}{w_0^2+w_1^2} \\
		\to y &= \frac{-bw_0w_1}{w_0\left(w_0^2+w_1^2\right)} \\
		\to y &= \frac{-bw_1}{\left(w_0^2+w_1^2\right)} \\ \\
		&\textrm{Thus we have our point of intercept, } \\
		I &= \left(\begin{matrix}
				\frac{-bw_0}{w_0^2+w_1^2} & \frac{-bw_1}{w_0^2+w_1^2}
			\end{matrix}\right)
	\end{split}\end{equation}

And now, for the cool bit. If you thought the rest of this was cool, the coolest bit is yet to come.

You might be thinking, `But Nathan, how does this coalesce into the answer?`, just you wait... just you wait.
This final bit is one of those moments that made me actually \textit{gasp} in excitement when I saw it. Because the
equation itself looks very scary, but its such an elegant simplification.

We now calculate $\norm{I}$, the euclidean norm of our intercept point $I$.

\begin{equation}\begin{split}
		\norm{I} &= \sqrt{
			\left(\frac{-bw_0}{w_0^2+w_1^2}\right)^2+\left(\frac{-bw_1}{w_0^2+w_1^2}\right)^2
		} \\
		\norm{I} &= \sqrt{
			\frac{b^2w_0^2}{\left(w_0^2+w_1^2\right)^2}+\frac{b^2w_1^2}{\left(w_0^2+w_1^2\right)^2}
		} \\
		\norm{I} &= \sqrt{
			\frac{b^2w_0^2+b^2w_1^2}{\left(w_0^2+w_1^2\right)^2}
		} \\
		\norm{I} &= \sqrt{
			\frac{b^2\left(w_0^2+w_1^2\right)}{\left(w_0^2+w_1^2\right)^2}
		} \\
		\norm{I} &= \sqrt{
			\frac{b^2}{w_0^2+w_1^2}
		} \\
		\norm{I} &= \frac{b}{\sqrt{w_0^2+w_1^2}} \\
		\norm{\mathbf{w}}=\sqrt{w_0^2+w_1^2} \therefore \norm{I} &= \frac{b}{\norm{\mathbf{w}}}
	\end{split}\end{equation}

And we have our answer! So cool.
\end{document}
