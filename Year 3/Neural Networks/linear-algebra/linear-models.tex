\subsection{Linear models}
A linear model describes a relationship between one variable and another.
The most common linear model you will be familiar with is:

\begin{equation}\label{straight-line}
	y = mx + c
\end{equation}

In GCSE Maths, this is referred to as \textit{the equation of a straight line}. It describes a
relationship between two variables, in this case $x$ and $y$, by using two constants, $m$, and $c$.
i.e. we are predicting the behaviour of the variable $y$ as $x$ changes.

There are of course, many many other kinds of linear model, the only real constraint is that they describe
a relationship between two variables.

A more formal notation:

\begin{equation}
	\hat{f}(x) = w_0+w_1x
\end{equation}

This describes the same model as \autoref{straight-line}, simply in terms of a function $\hat{f}$.
