\documentclass[a4paper]{article}
\usepackage{amsmath}
\usepackage{amsfonts}
\usepackage{hyperref}
\hypersetup{
    colorlinks=true,
    linkcolor=blue,
    filecolor=magenta,      
    urlcolor=cyan,
    pdftitle={Overleaf Example},
    pdfpagemode=FullScreen,
    }

\usepackage{listings}
\usepackage{color}
\definecolor{dkgreen}{rgb}{0, 0.6, 0}
\definecolor{gray}{rgb}{0.5, 0.5, 0.5}
\definecolor{mauve}{rgb}{0.58, 0, 0.82}

\lstset{frame=tb,
	language=Haskell,
	aboveskip=3mm,
  	belowskip=3mm,
  	showstringspaces=false,
  	columns=flexible,
  	basicstyle={\small\ttfamily},
  	numbers=none,
  	numberstyle=\tiny\color{gray},
  	keywordstyle=\color{blue},
 	 commentstyle=\color{dkgreen},
  	stringstyle=\color{mauve},
  	breaklines=true,
  	breakatwhitespace=true,
  	tabsize=3
}

\title{G6021 --- Comparative Programming}
\author{Dr. Vincent Van Oostrom \and Notes by Nathan Baines}
\date{Academic year 2025/26}

\begin{document}
\maketitle

\section{What is this module about?}
This module is concerned with conceptualising a comparison between the four different programming
paradigms.

These paradigms are explained in the following subsections. It is worth noting that a language can be
multi-paradigm, thus being capable of expressing a program in two or more of the four paradigms. 
Furthermore, if languages $A$ and $B$ are said to be \textit{Turing complete}, one might assume that any program expressed in $A$, can be translated to a program in $B$ which responds to inputs in the same way.


\subsection{Imperative}
The Imperative paradigm is based on the idea of a Turing machine, the idea is that every line implies a
particular instruction to be carried out.


\subsection{Object Oriented}
This paradigm is based on the concept of objects which can modify each other. By comparison, Imperative programming might be thought of as an Object Oriented program which only operates within a singular
object, which OOP allows for the existence of multiple objects, which can allow modify each other.


\subsection{Functional}
This paradigm is based on the use of functions, and function composition. A program is comprised of
many functions which are combined together through a sequence of funciton compositions, eventually
achieving the desired output.


\subsection{Logic}
The logical paradigm is arguably the least common, an example of it is in the language \textit{Prolog}.

The idea is that a problem is proposed using logical propositions, and the answer determined using
further supplied predicates. The aim of a program which follows this paradigm is to uncover the truths
implied by the supplied predicates.

\section{Functional Programming}
Functional programs consist entirely of functions, and built through function composition.
Expressions built by functions are then evaluated through reduction steps.

The focus of a functional programming language is describing \textit{what} is to be computed, rather than what is to be computed.

Functional programming languages are generally strongly typed, and have built-in memory
management. The advantage of their use is that programs are often shorter, and easier
to understand. They can also be easier to design and maintain that imperative programs,
but can suffer in terms of performance.

\subsection{Haskell}
Haskell is a functional programming language we will be using in this module. It features
a syntax which allows the definition of functions using mathematical-like syntax:

\begin{lstlisting}
square x = x * x
\end{lstlisting}

The role of the computer is to evaluate and display the results of expressions that the
programmer forms by composition of various functions, effectively like a sophisticated
calculator.

Such expressions are evaluated through a series of \textit{reduction steps}, for example,
using the above function, the following expression:

\begin{lstlisting}
square 6
\end{lstlisting}

Would be evaluated as follows:

\begin{equation}
square 6 \to 6\times6 \to 36
\end{equation}

\ldots finding the result of $36$. The \textit{complexity} of this expression is set as
$2$ then, since the total number of reduction steps required to evaluate this expression is $2$.
Note that the expression \textit{normal form}, refers to the form of an expression which
cannot be reduced any further.


\subsection{Reduction Sequences}
In pure functional languages the value of an expression is uniquely determined by its
components, and is independent of the order of reduction, this is the Unicity of normal
forms and provides the advantage of increased readability in programs. Some reduction
sequences however, might not lead to a value, some do not terminate.

The function $fortytwo(x) = 42$, and $infinity(x)=infinity(x)$ for example.

Some reduction sequences of the expression $fortytwo(infinity(x))$ will lead to $42$, but
some will not terminate, this expression has \textit{Unicity in its normal forms}, since
all the sequences which evaluate fully lead to $42$.

\subsubsection{Strategies}
Although the normal form may be unique, the order of reductions is important. The
\textit{Strategy of evaluation} defines the reduction sequence that a language implements.

The two most popular strategies are:
\begin{itemize}
	\item Call-by-name (Normal order): First reduce the application using the definition of the function, and then the argument.
	\item Call-by-value (Applicative order): First evaluate the argument, and then the application using the definition of the function.
\end{itemize}

Different strategies may result in a different number of reductions steps, and take different
paths through the expression's reduction graph.

Call-by-name \textit{always} finds a value, if there is one, where Call-by-value is generally
more efficient, but may fail to find a value.

Haskell uses a strategy called \textit{lazy} evaluation, which guarantees that if an 
expression has a normal form, the evaluator will find it. Lazy evaluation = call-by-name + sharing.

\subsection{Reduction Graphs}
A reduction graph illustrates \textit{all the possible ways an expression might be evaluated}.

The expression $square (3+4)$ for example, can be reduced multiple different ways, one might
first substitute $(3+4)$ into $square$, and then calculate $3+4$ twice, substituting again,
arriving at $7\times 7$. One might also calculate $3+4$, and substitute to $square(7)$, 
then leading to $7\times 7$.

\subsection{Functional Values}
Functions are also values. A function is simply a mapping which associates to each element
in its domain $A$, to the codomain $B$.

\begin{equation}
f:A\to B
\end{equation}

If a function of type $A\to B$ is applied to an argument $x\in A$, it will produce a result
which is a member of $B$, i.e. $f(x)\in B$.

In Haskell, the type of a function is describe as a sequence of the types involved in
its reduction steps.


\end{document}
