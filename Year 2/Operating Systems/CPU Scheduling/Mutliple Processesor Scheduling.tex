\subsection{Multiple Processor Scheduling}


\subsubsection{Symmetric Multiprocessing}

\begin{itemize}
    \item Processor is self-scheduling.
    \item All threads may be in common ready queues.
    \item Or, each processor may have its' own private queue of threads.
\end{itemize}


\subsubsection{Logical processors}

Each core may have multiple hardware threads, each forming a logical
processor, which from the OS' point of view is a separate processor.

The CPU may then switch between hardware threads when one thread is busy
waiting for memory, or some other event.


\subsubsection{Multithreaded Multicore Systems}

Intel Hyperthreading refers to a system in which each processor core has
multiple hardware threads.

This leads to two scheduler levels:
\begin{itemize}
    \item OS decides software thread which runs on each hardware thread.
    \item Each core decides which hardware thread runs on each physical processor core.
\end{itemize}


\subsubsection{Load balancing}

In \textit{Symmetric Multiprocessing}, all CPUs must be loaded for efficiency. Load
balancing keeps the workload between CPUs even.

\begin{itemize}
    \item Push migration --- Periodically checks load on CPUs and pushes tasks from overloaded CPUs to others.
    \item Pull migration --- Idle processors can pull waiting tasks from busy processors.
\end{itemize}

\subsubsection{Processor Affinity}

The data most recently accessed by a thread on a processor is stored in the processor's cache.
This is called \textit{Processor Affinity}, and may be affected by Load Balancing.

\begin{itemize}
    \item \textit{Soft Affinity} --- OS attempts to keep a thread on the same processor, but this cannot be guaranteed.
    \item \textit{Hard Affinity} --- Allows a process to specify a set of processes it may run on.
\end{itemize}


\subsubsection{NUMA Aware OS}

If an OS is \textit{NUMA Aware}, it will assign memory close to the CPU on which the process is running on.