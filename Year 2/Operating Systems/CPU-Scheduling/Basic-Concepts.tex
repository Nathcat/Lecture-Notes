\subsubsection{Basic Concepts}

Main concern of a scheduler is the distribution of CPU bursts.\
\begin{itemize}
    \item Large \# of short CPU bursts
    \item Small \# of large CPU bursts
\end{itemize}


Scheduler decisions may occur when a process:
\begin{enumerate}
    \item Switches from running to waiting
    \item Switches from running to ready
    \item Switches from waiting to ready
    \item Terminates
\end{enumerate}

When decisions are made at 1.\ or 4., the scheduler is
non-preemptive, otherwise it is preemptive.

A non-preemptive scheduler allows a process to keep control of the CPU
until it releases, either by terminating or switching to the waiting state.
A preemptive scheduler can switch a process out of the running state without termination.

A \textit{Race condition} is where the output of a program is dependent on the timing
or sequence of execution. Preemptive scheduling can lead to race conditions, where there is shared 
data between process / threads. One process may change data when the 2nd is preempted into the running state.
The 2nd process then reads from inconsistent data.

The \textit{Dispatcher} controls when process are switched into the CPU.
It controls:
\begin{itemize}
    \item Context switching
    \item Switching to user mode
    \item Restarting a user program from the correct point
\end{itemize}

\textit{Dispatch latency} is the time it takes the dispatcher to stop one
process and start another.