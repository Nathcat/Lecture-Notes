\subsection{Operations on Processes}


\subsubsection{Context Switches}

CPU undergoes a context switch to change active processes.

Current process state (PCB) is saved, and new process state is loaded.

The time this takes is pure overhead, and is completely dependent
on the hardware support for this process. Some CPUs for example
have multiple sets of registers, allowing them to have multiple contexts
loaded at the same time.


\subsubsection{Process Creation}

Parent processes can create child processes. Every process has an ID (PID) which
it managed through.

There are different options for how parent and child processes can share resources (or not):
\begin{itemize}
    \item Parent and child share all resources.
    \item Children share subset of parent's resources.
    \item Parent and child share no resources.
\end{itemize}

There are also options for how processes can be executed:
\begin{itemize}
    \item Parent and child are executed concurrently.
    \item Parent waits for child to terminate.
\end{itemize}

The child process can be a duplicate of the parent or
it can have a new program loaded into it.
\begin{itemize}
    \item \textit{fork} --- Creates new process
    \item \textit{exec} --- Syscall after fork to replace processes' memory with a new program.
    \item \textit{wait} --- Lets parent wait for child termination. Provides child state information to parent.
    \item \textit{exit} --- Terminates process and returns state information to parent.
    \item \textit{abort} --- Called by parent to terminate a child process.
\end{itemize}

The OS may or may not allow child processes to exist without a parent.
If it does not, it will use \textit{cascaded termination} to terminate all child processes
when their parent is terminated.

If a child terminates and the parent has already terminated without calling wait, the child
is a \textit{Zombie process}.

If the parent of a child terminates without calling wait, the children of that process
are \textit{Orphan processes}.