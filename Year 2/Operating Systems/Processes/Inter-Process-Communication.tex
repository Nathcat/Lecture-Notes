\subsection{Inter-Process Communication --- IPC}

There are two widely accepted models of IPC.\@
\begin{itemize}
    \item Shared memory
    \item Message passing
\end{itemize}


\subsubsection{Shared Memory}

Communications are under the control of the processes involved.
A major issue with this model is synchronisation between the processes.


\subsubsection{Message Passing}

The OS provides methods to send and receive messages between processes.

This model involves no shared resources, but some form of communication link
must be established between processes.


\subsubsection{Message Passing --- Direct Communication}

\begin{itemize}
    \item $send(P, m)$ --- $m$ is sent to process $P$.
    \item $receive(Q, m)$ --- $m$ is received from processes $Q$.
\end{itemize}

Communication link is established between one pair of processes
automatically. Between each pair of processes there exists exactly
one link. This link can be both uni-directional or bi-directional.


\subsubsection{Message Passing --- Indirect Communication}

Messages are directed and received through mailboxes. Each mailbox has
a unique ID.\ Processes can only communicate if they share a mailbox.

The mailboxes here form the communication link, and can be shared between
multiple processes. Each pair of processes may share several communication
links. This link may be either uni-directional or bi-directional.

The OS provides operations to:
\begin{itemize}
    \item Create mailboxes (ports).
    \item $send(A, m)$ --- Send $m$ to mailbox $A$.
    \item $receive(A, m)$ --- Receive $m$ from mailbox $A$.
    \item Delete mailboxes.
\end{itemize}

The main issue with a shared link such as this is determining which
process receives a messsage. There are three main solutions to this:
\begin{itemize}
    \item Allowing a link to be associated with only two processes.
    \item Allow one process at a time to receive.
    \item Arbitrarily select receiver, sender is then notified of who received their message.
\end{itemize}


\subsubsection{Message Passing --- Buffering}

A queue of messages is maintained on each link.
\begin{itemize}
    \item Zero capacity --- No queueing, sender must wait for the receiver.
    \item Bounded capacity --- Finite queue length, when full, the sender must wait for the receiver.
    \item Unbounded capacity --- Infinite queue length, the sender never needs to wait.
\end{itemize}