\subsection{Concept}

\subsubsection{What is a process}

A \textit{Process} is a \textit{Program} in execution. A Program then is a passive entity
on the physical storage disk. A Process is a Program in execution. Note that a Program might
contain multiple Processes. A Process is an active entity.


\subsubsection{Memory Layout}

The layout of a program in memory is as follows:

\begin{enumerate}
    \item Stack --- Temporary, local scope data (params, local variables).
    \item Heap --- Dynamically allocated memory (e.g. Malloc).
    \item Data --- Global variables
    \item Text --- Executable code
\end{enumerate}

There is sufficient space between the Stack and Heap sections to allow them to expand.


\subsubsection{State}

\begin{itemize}
    \item New --- Being created
    \item Running
    \item Waiting --- Waiting for the result of some event
    \item Ready --- Waiting to be allocated to a process
    \item Terminated
\end{itemize}


\subsubsection{Process Control Block --- PCB}

\begin{itemize}
    \item State
    \item Number
    \item Program Counter
    \item CPU Register contents
    \item Memory info --- Memory limits, list of open files, etc\ldots
    \item Accounting info --- CPU used, Clock time elapsed since start.
\end{itemize}


\subsubsection{Threads}

Threads are facilitated through the use of multiple program counters in the PCB.

This allows multiple locations in the code to be executed at once.
The effect is the outward appearance of multiple processes which share
resources, and are able to exploit concurrency.