\subsection{Critical Section}

The \textit{Critical Section --- CS} of a program is a section of code in which
a program is manipulating some shared memory between proceses. The problem with
this process lies in synchronising these two processes to ensure the data they
use is valid and consistent.

In order to ensure this, the CS of a process must be mutually exclusive, i.e.\ 
when one program is in its' CS, no other program can be in its' CS.


\subsubsection{Solution Requirements}

\begin{itemize}
    \item Mutually exclusive
    \item Progress --- If no process is in its' CS, the selection of the next process to enter its' CS cannot wait indefinitely.
    \item Bounded Waiting --- Some bound exists on the \# of programs which can enter their CS after another program has requested to enter its' CS, before this request is granted.
\end{itemize}


\subsubsection{Interrupt Solution}

On the entry section (before the CS), disable interrupts, and on the exit section (leaving the CS), enable interrupts.

This ensures that the process will complete its' CS, but in the case that the CS is very long, other programs must
wait a long time to enter their own CS (\textit{Starvation}). Hence this approach can be innefficient in multiprocessor
systems.

\subsubsection{Peterson's Solution}

This solution works to synchronise two processes. It assumes that the $load$ and $store$ instructions are atomic and cannot
be interrupted.

The two processes in question share two variables:
\begin{itemize}
    \item int turn --- Defines which process' turn it is to enter their CS.
    \item boolean flag[2] --- Indicates if each process is ready to enter their CS.
\end{itemize}

\begin{equation}
    \begin{split}
        &\quad\quad Proc\quad P_i \\
        &flag_i\to 1 \\
        &turn\to j \\
        &while\quad(flag_j=1); \\
        &\{CS\} \\
        &flag_i\to 0 \\
        &\{remainder\}
    \end{split}
\end{equation}

The processes may either interfere with each other, or not interfere:
\begin{itemize}
    \item Non-interfering --- If $P_i$ never goes to entry, $P_j$ will simply run to completion.
    \item Interfering --- If both $P_i$ and $P_j$ go to entry, the last to set $turn$ will stall at entry until the other is complete.
\end{itemize}

This solution may not work in modern systems, since they can reorder instructions which are not
dependent on each other to improve efficiency. This may result in an unexpected order of instructions,
and therefore unexpected behaviour with this approach.

A fix to this is to use a \textit{Memory Barrier}, when used, the system will then ensure that all $load$
and $store$ instructions are completed before proceeding, effectively ensuring the order or instructions will be
followed as intended.