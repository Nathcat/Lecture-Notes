\subsection{Multicore Programming}

\textit{Parrallelism} --- Implies a system which can perform multiple tasks simultaneously.

\textit{Concurrency} --- Supports more than one task making progress. A scheduler may provide concurrency on single core systems.

Note that there is a distinction between \textit{User threads} --- Threads created and managed with a
user level library, and \textit{Kernel threads} --- Threads created and run as part of the Kernel.
\subsubsection{Amdahl's Law}

Describes the performance gains from adding additional processing cores
to an application with both serial and parrallel components.

\begin{equation}
    \begin{split}
        &S = \textrm{Serial portion},S\in[0, 1] \\
        &N=\textrm{\# of cores} \\
        \\
        &speedup\leq\frac{1}{S+\frac{1-S}{N}}
    \end{split}
\end{equation}

This leads us to

\begin{equation}
    \lim_{N\to\infty}speedup = \frac{1}{S}
\end{equation}

Hence, the serial portion of the application has a disproportional effect
on the performance gain from adding additional cores.
