\documentclass[a4paper]{report}
\usepackage{amsmath}
\usepackage{amsfonts}
\usepackage{listings}
\usepackage{color}
\usepackage{xcolor}
\usepackage[T1]{fontenc}


\definecolor{dkgreen}{rgb}{0,0.6,0}
\definecolor{gray}{rgb}{0.5,0.5,0.5}
\definecolor{mauve}{rgb}{0.58,0,0.82}

\definecolor{delim}{RGB}{20,105,176}
\definecolor{numb}{RGB}{106, 109, 32}
\definecolor{string}{rgb}{0.64,0.08,0.08}

\lstdefinelanguage{json}{
    numbers=left,
    numberstyle=\small,
    frame=single,
    rulecolor=\color{black},
    showspaces=false,
    showtabs=false,
    breaklines=true,
    postbreak=\raisebox{0ex}[0ex][0ex]{\ensuremath{\color{gray}\hookrightarrow\space}},
    breakatwhitespace=true,
    basicstyle=\ttfamily\small,
    upquote=true,
    morestring=[b]",
    stringstyle=\color{string},
    literate=
     *{0}{{{\color{numb}0}}}{1}
      {1}{{{\color{numb}1}}}{1}
      {2}{{{\color{numb}2}}}{1}
      {3}{{{\color{numb}3}}}{1}
      {4}{{{\color{numb}4}}}{1}
      {5}{{{\color{numb}5}}}{1}
      {6}{{{\color{numb}6}}}{1}
      {7}{{{\color{numb}7}}}{1}
      {8}{{{\color{numb}8}}}{1}
      {9}{{{\color{numb}9}}}{1}
      {\{}{{{\color{delim}{\{}}}}{1}
      {\}}{{{\color{delim}{\}}}}}{1}
      {[}{{{\color{delim}{[}}}}{1}
      {]}{{{\color{delim}{]}}}}{1},
}

\lstset{frame=tb,
  language=Java,
  aboveskip=3mm,
  belowskip=3mm,
  showstringspaces=false,
  columns=flexible,
  basicstyle={\small\ttfamily},
  numbers=none,
  numberstyle=\tiny\color{gray},
  keywordstyle=\color{blue},
  commentstyle=\color{dkgreen},
  stringstyle=\color{mauve},
  breaklines=true,
  breakatwhitespace=true,
  tabsize=3
}


\title{G6046 --- Software Engineering --- Exam}
\author{Candidate \#279103}
\date{Academic year 2024/25}

\begin{document}
\maketitle

\section{Question 1}


\subsection{Part A}

Mandatory functional requirements:
\begin{itemize}
    \item System for users to create and authenticate accounts
    \item Conference system (ability to host / join conferences)
\end{itemize}

Desirable functional requirements:
\begin{itemize}
    \item Ability to store conference recordings
    \item Cloud storage system
\end{itemize}


\subsection{Part B}
Mandatory non-functional requirements:
\begin{itemize}
    \item Audio $+$ video suppoort in conferences
    \item Users under 18 not to appear in search results.
\end{itemize}

Desirable non-functional requirements:
\begin{itemize}
    \item Uploadable picture to be between 400$\times$400 and 1000$\times$1000 pixels.
    \item Mean time between failures of close to 500 hours
\end{itemize}


\subsection{Part C}
Code Red have stated that \textit{TaskWorld} will allow users under the
age of 18 to use thei platform, albeit with certain restrictions. As such a domain
requirement might be that the application has safety features to support such a user base.

In order to implement this they would need to ensure that TaskWorld strictly confirms to
data protection and safety laws, and laws specific to child safety. They would need to implement
robust features that allow their users to have direct control over the data Code Red are holding
about them, and the kind of content they might be exposed to on the platform. They might, for example,
add a \textit{blocking} feature to allows users to block other users, or a \textit{reporting} feature
to allow users to report the actions of other users for review by some moderation system.

These measures would help improve the safety of the platform for the stated user base, and help protect
Code Red from potential legal action in the future.


\subsection{Part D}
Firstly, the young team may indicate a lack of experience in big projects like this. This might hinder
progress as they learn new technologies and how to work together, and potentially
decrease the overall quality of the final product. This is a \textit{Lack of Skilled Personnel}

The team does however, have Chris, a veteran with lots of experience in projects like this. Notably though,
\textit{Chris only accepted the job as a favour to a member of the board he is good friends with}, this might
indicate a lack of investment or interest in the project from Chris. Perhaps he will not give his full effort
into assisting the team, or leave the team completely, because it is not a project he finds interesting.
This effectively leaves the young and potentially inexperienced team to their own devices, which as previously 
discussed, might hinder progress. This is an issue of \textit{Employee Turnover}


\subsection{Part E}
As stated, the TaskWorld team are very keen to shape the development process, and with large projects like this
it is essential to get client feedback to ensure that each part of the application fits the expectations.
As such, Agile is a good fit for this project.

The Agile methodology expects that each part of the application is developed in so-called \textit{Sprints}, and
then presented to the client for feedback, and continually improved until the client accepts it. Each Sprint adding new
features until the client is happy with the final product. The notable part here is that the client is heavily involved,
after every sprint they are consulted to gain feedback on the current system, which is then improved using their feedback.

Hence, using the Agile methodology would be a good fit here, as it would allow the TaskWorld team (the client) to provide 
their feedback at each stage of development, helping to shape the development process. Furthermore, since they are keen to
be involved, we can assume their feedback will be meaningful and helpful, which further helps to enhance the final product.

Agile also tends to produce good documentation, as each part of the application can be documented as it is created. This helps
lead to a good quality, maintainable codebase. If the development team expands in the future as the application grows, this kind
of maintainability and good documentation will be invaluable to improve and build off the older systems built into the codebase
by the original team.


\section{Question 3}


\subsection{Part A}

\begin{lstlisting}
    public class Product
\end{lstlisting}

\subsubsection{Fields}
\begin{itemize}
    \item Name --- String --- The name of the product
    \item Price --- Float --- The price of the product (potentially also Integer)
    \item Supplier --- Integer --- The Numeric code for the supplier of this product
    \item ProductCode --- Integer --- The Suppler Product Code for this product.
\end{itemize}


\begin{lstlisting}
    public class BulbPacket extends Product
\end{lstlisting}

\subsubsection{Fields}
\begin{itemize}
    \item NoInPacket --- Integer --- The number of bulbs in the packet.
    \item MonthsToDevelop --- Integer --- The number of months these bulbs take to develop.
\end{itemize}


\begin{lstlisting}
    public class LivePlant extends Product
\end{lstlisting}

\subsubsection{Fields}
\begin{itemize}
    \item LatinName --- String --- The Latin name of this plant.
    \item IsIndoor --- Boolean --- Whether or not this is an indoor plant. If false it is an outdoor plant.
\end{itemize}


\begin{lstlisting}
    public class Statue extends Product
\end{lstlisting}

\subsubsection{Fields}
\begin{itemize}
    \item Height --- Integer --- The height of this Statue in $cm$.
    \item Material --- String --- The Material this Statue is made of (potentially an enum of potential materials as well)
    \item Fragile --- Boolean --- Whether or not this Statue is fragile.
\end{itemize}


\begin{lstlisting}
    public class Customer
\end{lstlisting}

Note that to facilitate fast lookups by a customer's last name, there will need to be some other system
implemented which can do this. You might link the hash of a customer's last name to the instance describing that
customer's information, for example.

\subsubsection{Fields}
\begin{itemize}
    \item FirstName --- String --- The Customer's first name.
    \item LastName --- String --- The Customer's last name. Separated to facilitate faster lookups by last name.
    \item Address --- String --- The Customer's address.
    \item Telephone --- String --- The Customer's Telephone number.
    \item EmailAddress --- String --- The Customer's Email address.
    \item Orders --- Array of Order --- A list of orders placed by this customer.
\end{itemize}

\subsubsection{Methods}
\begin{lstlisting}
    public void placeOrder(Item[])
\end{lstlisting}

Place an order for the given list of items.


\begin{lstlisting}
    public class Order
\end{lstlisting}

\subsubsection{Fields}
\begin{itemize}
    \item Number --- Integer --- The unique order number
    \item Products --- Product[] --- The products in this order
    \item Status --- String --- The current status of this order
\end{itemize}


\subsection{Part B}
\begin{lstlisting}
    public class ProductReviewTest {
        @Test
        public void test1() {
            String author = "John Smith";
            String text = "Great product";
            int rating = 5;
            ProductReview pr = new ProductReview(author, text, rating);

            assertEquals(author, pr.getAuthor());
            assertEquals(text, pr.getReviewText());
            assertEquals(0, pr.getVotes());

            pr.upVote();
            assertEquals(1, pr.getVotes());

            pr.downVote(); pr.downVote();
            assertEquals(0, pr.getVotes());

            pr.markAsSpam();
            assertEquals("Marked as Spam!", pr.getReviewText());
            
            pr.markAsNotSpam();
            for (int i = 0; i < 150; i++) pr.upVote();
            assertEquals(100, pr.getVotes());

            for (int i = 0; i < 150; i++) pr.downVote();
            assertEquals(0, pr.getVotes());

            pr = new ProductReview(author, "Fab Book", 10);
            assertEquals(0, pr.getRating());
        }
    }
\end{lstlisting}


\subsection{Part C}

\subsubsection{XML}
\lstset{language=XML,morekeywords={
    event, 
    date, 
    weekday,
    day,
    month,
    year, 
    meeting,
    name,
    time,
    is24HoursTime,
    hours,
    minutes,
    duration,
    location,
    isOnline,
    participants, 
    participant
    }}

\begin{lstlisting}
    <event>
        <date>
            <weekday>Friday</weekday>
            <day>1</day>
            <month>12</month>
            <year>2023</year>
        </date>

        <meeting>
            <name>Management meeting</name>
            <time>
                <is24HoursTime>true</is24HoursTime>
                <hours>14</hours>
                <minutes>30</minutes>
            </time>
            <duration>60</duration>
        </meeting>

        <location>
            <isOnline>false</isOnline>
            <name>room MB101</name>
        </location>

        <participants>
            <participant>David Thewlis</participant>
            <participant>Amy Pond</participant>
            <participant>Arthur Lowe</participant>
        </participants>
    </event>
\end{lstlisting}

\subsubsection{JSON}
\lstset{language=json}

\begin{lstlisting}
    {
        "date": {
            "weekday": "Friday",
            "day": 1,
            "month": 12,
            "year": 2023
        },

        "meeting": {
            "name": "Management meeting",
            "time": {
                "is24HoursTime": true,
                "hours": 14,
                "minutes": 30
            },
            "duration": 60
        },

        "location": {
            "isOnline": false,
            "name": "room MB101"
        },

        "participants": [
            {
                "name": "David Thewlis"
            },
            {
                "name": "Amy Pond",
            },
            {
                "name": "Arthur Lowe"
            }
        ]
    }
\end{lstlisting}
\end{document}